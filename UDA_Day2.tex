% Options for packages loaded elsewhere
\PassOptionsToPackage{unicode}{hyperref}
\PassOptionsToPackage{hyphens}{url}
%
\documentclass[
]{article}
\usepackage{amsmath,amssymb}
\usepackage{lmodern}
\usepackage{iftex}
\ifPDFTeX
  \usepackage[T1]{fontenc}
  \usepackage[utf8]{inputenc}
  \usepackage{textcomp} % provide euro and other symbols
\else % if luatex or xetex
  \usepackage{unicode-math}
  \defaultfontfeatures{Scale=MatchLowercase}
  \defaultfontfeatures[\rmfamily]{Ligatures=TeX,Scale=1}
\fi
% Use upquote if available, for straight quotes in verbatim environments
\IfFileExists{upquote.sty}{\usepackage{upquote}}{}
\IfFileExists{microtype.sty}{% use microtype if available
  \usepackage[]{microtype}
  \UseMicrotypeSet[protrusion]{basicmath} % disable protrusion for tt fonts
}{}
\makeatletter
\@ifundefined{KOMAClassName}{% if non-KOMA class
  \IfFileExists{parskip.sty}{%
    \usepackage{parskip}
  }{% else
    \setlength{\parindent}{0pt}
    \setlength{\parskip}{6pt plus 2pt minus 1pt}}
}{% if KOMA class
  \KOMAoptions{parskip=half}}
\makeatother
\usepackage{xcolor}
\usepackage[margin=1in]{geometry}
\usepackage{color}
\usepackage{fancyvrb}
\newcommand{\VerbBar}{|}
\newcommand{\VERB}{\Verb[commandchars=\\\{\}]}
\DefineVerbatimEnvironment{Highlighting}{Verbatim}{commandchars=\\\{\}}
% Add ',fontsize=\small' for more characters per line
\usepackage{framed}
\definecolor{shadecolor}{RGB}{248,248,248}
\newenvironment{Shaded}{\begin{snugshade}}{\end{snugshade}}
\newcommand{\AlertTok}[1]{\textcolor[rgb]{0.94,0.16,0.16}{#1}}
\newcommand{\AnnotationTok}[1]{\textcolor[rgb]{0.56,0.35,0.01}{\textbf{\textit{#1}}}}
\newcommand{\AttributeTok}[1]{\textcolor[rgb]{0.77,0.63,0.00}{#1}}
\newcommand{\BaseNTok}[1]{\textcolor[rgb]{0.00,0.00,0.81}{#1}}
\newcommand{\BuiltInTok}[1]{#1}
\newcommand{\CharTok}[1]{\textcolor[rgb]{0.31,0.60,0.02}{#1}}
\newcommand{\CommentTok}[1]{\textcolor[rgb]{0.56,0.35,0.01}{\textit{#1}}}
\newcommand{\CommentVarTok}[1]{\textcolor[rgb]{0.56,0.35,0.01}{\textbf{\textit{#1}}}}
\newcommand{\ConstantTok}[1]{\textcolor[rgb]{0.00,0.00,0.00}{#1}}
\newcommand{\ControlFlowTok}[1]{\textcolor[rgb]{0.13,0.29,0.53}{\textbf{#1}}}
\newcommand{\DataTypeTok}[1]{\textcolor[rgb]{0.13,0.29,0.53}{#1}}
\newcommand{\DecValTok}[1]{\textcolor[rgb]{0.00,0.00,0.81}{#1}}
\newcommand{\DocumentationTok}[1]{\textcolor[rgb]{0.56,0.35,0.01}{\textbf{\textit{#1}}}}
\newcommand{\ErrorTok}[1]{\textcolor[rgb]{0.64,0.00,0.00}{\textbf{#1}}}
\newcommand{\ExtensionTok}[1]{#1}
\newcommand{\FloatTok}[1]{\textcolor[rgb]{0.00,0.00,0.81}{#1}}
\newcommand{\FunctionTok}[1]{\textcolor[rgb]{0.00,0.00,0.00}{#1}}
\newcommand{\ImportTok}[1]{#1}
\newcommand{\InformationTok}[1]{\textcolor[rgb]{0.56,0.35,0.01}{\textbf{\textit{#1}}}}
\newcommand{\KeywordTok}[1]{\textcolor[rgb]{0.13,0.29,0.53}{\textbf{#1}}}
\newcommand{\NormalTok}[1]{#1}
\newcommand{\OperatorTok}[1]{\textcolor[rgb]{0.81,0.36,0.00}{\textbf{#1}}}
\newcommand{\OtherTok}[1]{\textcolor[rgb]{0.56,0.35,0.01}{#1}}
\newcommand{\PreprocessorTok}[1]{\textcolor[rgb]{0.56,0.35,0.01}{\textit{#1}}}
\newcommand{\RegionMarkerTok}[1]{#1}
\newcommand{\SpecialCharTok}[1]{\textcolor[rgb]{0.00,0.00,0.00}{#1}}
\newcommand{\SpecialStringTok}[1]{\textcolor[rgb]{0.31,0.60,0.02}{#1}}
\newcommand{\StringTok}[1]{\textcolor[rgb]{0.31,0.60,0.02}{#1}}
\newcommand{\VariableTok}[1]{\textcolor[rgb]{0.00,0.00,0.00}{#1}}
\newcommand{\VerbatimStringTok}[1]{\textcolor[rgb]{0.31,0.60,0.02}{#1}}
\newcommand{\WarningTok}[1]{\textcolor[rgb]{0.56,0.35,0.01}{\textbf{\textit{#1}}}}
\usepackage{graphicx}
\makeatletter
\def\maxwidth{\ifdim\Gin@nat@width>\linewidth\linewidth\else\Gin@nat@width\fi}
\def\maxheight{\ifdim\Gin@nat@height>\textheight\textheight\else\Gin@nat@height\fi}
\makeatother
% Scale images if necessary, so that they will not overflow the page
% margins by default, and it is still possible to overwrite the defaults
% using explicit options in \includegraphics[width, height, ...]{}
\setkeys{Gin}{width=\maxwidth,height=\maxheight,keepaspectratio}
% Set default figure placement to htbp
\makeatletter
\def\fps@figure{htbp}
\makeatother
\setlength{\emergencystretch}{3em} % prevent overfull lines
\providecommand{\tightlist}{%
  \setlength{\itemsep}{0pt}\setlength{\parskip}{0pt}}
\setcounter{secnumdepth}{-\maxdimen} % remove section numbering
\ifLuaTeX
  \usepackage{selnolig}  % disable illegal ligatures
\fi
\IfFileExists{bookmark.sty}{\usepackage{bookmark}}{\usepackage{hyperref}}
\IfFileExists{xurl.sty}{\usepackage{xurl}}{} % add URL line breaks if available
\urlstyle{same} % disable monospaced font for URLs
\hypersetup{
  pdftitle={Analyzing Text: Preliminaries},
  pdfauthor={Unstructured Data Analytics},
  hidelinks,
  pdfcreator={LaTeX via pandoc}}

\title{Analyzing Text: Preliminaries}
\author{Unstructured Data Analytics}
\date{}

\begin{document}
\maketitle

\hypertarget{well-well-well-what-do-we-have-here}{%
\section{Well well well, What do we have
here?}\label{well-well-well-what-do-we-have-here}}

\emph{First explorations of a new dataset}

\begin{Shaded}
\begin{Highlighting}[]
\FunctionTok{library}\NormalTok{(tidyverse)}
\FunctionTok{library}\NormalTok{(stringr)}
\end{Highlighting}
\end{Shaded}

\hypertarget{lets-read-in-some-data-and-check-it-out}{%
\subsection{Let's read in some data and check it
out}\label{lets-read-in-some-data-and-check-it-out}}

\begin{Shaded}
\begin{Highlighting}[]
\CommentTok{\#read it in}
\CommentTok{\#sports \textless{}{-} read\_csv("???/UDA Fall 2023 Stories.csv")}
\end{Highlighting}
\end{Shaded}

How can we explore this dataset to find out what it offers us?

\hypertarget{bag-of-words-methods}{%
\subsection{``Bag of words'' methods}\label{bag-of-words-methods}}

\emph{How do we find the words?} \emph{What issues might we run into?}

\hypertarget{dividing-and-counting}{%
\subsubsection{Dividing and Counting}\label{dividing-and-counting}}

Can you count how many words there are in one Story? In all the stories?
Can you tell me how many times a particular word shows up in one Story?
In all the stories?

\hypertarget{is-there-an-easier-way}{%
\subsection{Is there an easier way?}\label{is-there-an-easier-way}}

Yes, of course, this is R.

\begin{Shaded}
\begin{Highlighting}[]
\CommentTok{\#load package}
\end{Highlighting}
\end{Shaded}

Divide and prep

What changes were made from the starting text to the version?

\begin{Shaded}
\begin{Highlighting}[]
\CommentTok{\#write code to explore as needed}
\end{Highlighting}
\end{Shaded}

\hypertarget{counts-are-surprisingly-useful}{%
\subsection{Counts are surprisingly
useful}\label{counts-are-surprisingly-useful}}

\hypertarget{which-words-are-most-common-overall}{%
\subsubsection{Which words are most common
overall?}\label{which-words-are-most-common-overall}}

\hypertarget{which-words-are-most-common-in-each-story}{%
\subsubsection{Which words are most common in each
story?}\label{which-words-are-most-common-in-each-story}}

Can you create a top-five list of words for the whole file? Can you
create a top-five list of words for each document? \emph{Store these as
new objects called ``top\_by\_file'' and ``top\_by\_doc''}

\begin{Shaded}
\begin{Highlighting}[]
\CommentTok{\#everyone loves a "top" list}
\end{Highlighting}
\end{Shaded}

\hypertarget{can-we-get-a-better-list-of-common-words}{%
\subsection{Can we get a better list of common
words?}\label{can-we-get-a-better-list-of-common-words}}

Yes, of course.

\begin{Shaded}
\begin{Highlighting}[]
\CommentTok{\#load the data}
\end{Highlighting}
\end{Shaded}

How many lexicons are there? How many words are there in each lexicon?
\emph{Do you want more dictionary options? Try the ``stopwords''
package. It has additional stop words dictionaries, including support
for foreign languages.}

\begin{Shaded}
\begin{Highlighting}[]
\CommentTok{\#explore lexicons}
\end{Highlighting}
\end{Shaded}

What happens to our counts if we remove the stopwords using the SMART
lexicon?

\begin{Shaded}
\begin{Highlighting}[]
\CommentTok{\#remove SMART words}
\end{Highlighting}
\end{Shaded}

Can we see which stopwords were removed?\\
(Are there any we might want to keep?)

\begin{Shaded}
\begin{Highlighting}[]
\CommentTok{\#review what we lost}
\end{Highlighting}
\end{Shaded}


\end{document}
